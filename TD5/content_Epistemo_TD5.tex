
\newcommand{\numTD}{TD5}
\newcommand{\themeTD}{Complexité}

\begin{center}
\begin{tabular}{|p{2cm}c|}
\hline
{\includegraphics[width=1.8cm,viewport=0 0 337 248]{../images/sorbonne.png}} & \raisebox{2ex}{\begin{Large}\textbf{M1SOL020}\end{Large}}\\
2019-2020& \raisebox{2ex}{\begin{Large}\textbf{ Epistémologie de l'Informatique}\end{Large}}\\
&  \begin{large}\textbf{\numTD}\end{large} \\
&  \begin{large} \textbf{\themeTD}\end{large} \\
& Gaël Lejeune, Sorbonne Université \\
& \tiny{Inspiré d'Agnès Delaborde 2015-2016}\\
\hline
\end{tabular}
\end{center}


\hrule
%%%%%%%%%%%%%%%%%%%%%%%%%EN-TETE%%%%%%%%%%%%%%%%%%%%%%%%%%%%%
%\renewcommand{\contentsname}{Sommaire du TD}
%\tableofcontents

\noindent\fcolorbox{red}{lightgray}{
\begin{minipage}{12cm}
\section*{Objectifs}

\begin{itemize}
  \item Comprendre la notion de complexité
  \item En déduire l'importance de l'algorithme
  \item Parcourir un tableau de façon efficace
\end{itemize}
\end{minipage}
}

\section{Itérations}

\begin{itemize}
  \item  Rédiger l'algorithme (la séquence d'instructions) d'une boucle de 1 à 100.  A chaque pas de la boucle, une variable n prend la valeur du compteur. En sortie de boucle, on affiche n.
  \item  Réaliser le même algorithme pour une boucle de 1 à 100*1000
  \item  Codez ces deux algorithmes en \textsc{Python}
  \item  Mesurez la différence dans le temps d'exécution entre ces deux scripts : utilisez la méthode \texttt{clock} de la librairie \texttt{time} en stockant la valeur de \texttt{time.clock()} avant la boucle et en le comparant à sa valeur après la boucle.
  \item  Quelle est la complexité de l'algorithme (unité: nombre d'itérations) ?
\end{itemize}

\section{Parcours de tableau}

\begin{itemize}
  \item  Ecrivez un algorithme de création d'un tableau de 4 par 4. Chaque cellule contient une lettre, et chaque colonne représente l'alphabet (on s'arrête ici au bout de la 4ème lettre puisqu'il s'agit d'un tableau de 4 par 4).

\begin{tabular}{c|c|c|c}
A&A&A&A\\
\hline
B&B&B&B\\
\hline
C&C&C&C\\
\hline
D&D&D&D\\
\end{tabular}

  \item  Ecrivez un algorithme de parcours du tableau pour compter toutes les occurrences des lettres "A" et "C". Cet algorithme ne doit parcourir qu'une seule fois le tableau.
  \item  Rédigez le même algorithme, sauf que vous parcourrez le tableau une fois pour compter la présence de "A", et une seconde fois pour compter la présence des "C".
  \item  Quelle est la complexité de chacun des deux algorithmes (unité: nombre de passage dans la boucle) ?
  \item  Remarquez les temps d'exécution pour le(s) parcours du tableau. 
  \item  Si nous avions, au lieu d'un tableau 4 par 4, un tableau de 1000 par 2000, quelle serait la complexité des deux parcours ?
\end{itemize}

\section{Analyses de complexité}

 Quelle est la différence de complexité entre ces deux algorithmes qui vérifient la présence d'un entier dans un tableau (unité: nombre de passages dans la boucle tantque) ?

\vspace{0.5cm}

\textbf{Algorithme 1}
\vspace{-0.3cm}
\begin{verbatim}

e : entier; T: tableau [1…n] d'entiers
i <- 1; trouve <- FAUX
tantque trouve = FAUX et i <= n faire
    si e = T[i] alors
        trouve <- VRAI
    i <- i + 1
\end{verbatim}


\textbf{Algorithme 2}
\vspace{-0.3cm}

\begin{verbatim}
e : entier; T: tableau [1…n] d'entiers
i <- 1; trouve <- FAUX
tantque i <= n faire
    si e=T[i] alors
        trouve <- VRAI
        sortir de la boucle tantque
    i <- i + 1
\end{verbatim}
\vspace{-0.5cm}
\begin{itemize}
  \item  Indiquez, pour l'algorithme 1 ci-dessus, la complexité dans le meilleur des cas, dans le pire des cas, et la complexité moyenne.
  \item  Quelle est la complexité de l'algorithme ci-dessous ? (unité: passage dans la boucle)

\begin{python}
i <- 0; n <- 4; v <- 1;
tantque i <= n faire
    v <- v x 2
    afficher v
\end{python}
  \item  Quelle est la complexité de l'algorithme ci-dessous ? (nombre d'opérations effectuées)
\begin{python}
i <- 50;
y <- 4.0 / 3.0 * 4 * i
afficher y
\end{python}

  \item  Peut-on rendre plus efficace l'algorithme ci-dessous ?
\begin{verbatim}
fonction affichtab(tableau: liste d\'elements)
    i <- 0; n <- taille max de tableau
    tantque i <= n faire:
        afficher valeur de tableau à l'indice i
        i <- i + 1
s = "tableau d'entiers"
t: tableau d'entiers[1...10]
affiche "Voici le contenu de " et affiche s
appel a la fonction affichtab(t)
s = "chaîne de caractères"
t: tableau de caractères["a"..."z"]
affiche "Voici le contenu de " et affiche s
appel a la fonction affichtab(t)
\end{verbatim}
\end{itemize}
%f. Proposez un algorithme efficace pour parcourir le tableau ci-dessous. Le parcours doit s'arrêter lorsqu'on a trouvé la combinaison [3][4][2].
%\begin{python}
%t[entiers de 0 à 4][entiers de 0 à 7][entiers de 0 à 2]

%\end{python}
