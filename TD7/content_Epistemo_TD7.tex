
\newcommand{\numTD}{TD7}
\newcommand{\themeTD}{Annotation et Evaluation}

\begin{center}
\begin{tabular}{|p{2cm}c|}
\hline
{\includegraphics[width=1.8cm,viewport=0 0 337 248]{../images/sorbonne.png}} & \raisebox{2ex}{\begin{Large}\textbf{M1SOL020}\end{Large}}\\
2019-2020& \raisebox{2ex}{\begin{Large}\textbf{ Epistémologie de l'Informatique}\end{Large}}\\
&  \begin{large}\textbf{\numTD}\end{large} \\
&  \begin{large} \textbf{\themeTD}\end{large} \\
& Gaël Lejeune, Sorbonne Université \\
& \tiny{Inspiré d'Agnès Delaborde 2015-2016}\\
\hline
\end{tabular}
\end{center}


\hrule
%%%%%%%%%%%%%%%%%%%%%%%%%EN-TETE%%%%%%%%%%%%%%%%%%%%%%%%%%%%%
%\renewcommand{\contentsname}{Sommaire du TD}
%\tableofcontents

\noindent\fcolorbox{red}{lightgray}{
\begin{minipage}{12cm}
\section*{Objectifs}

\begin{itemize}
  \item Comprendre les enjeux de l'annotation d'un corpus
  \item Savoir évaluer l'annotation d'un corpus
\end{itemize}
\end{minipage}
}

%\textcolor{blue}{\textbf{Attention, le dernier exercice est à rendre sur Moodle pour le 08/12/2019}}

\section{Annotation linguistique}
%  \subsection{ Fouille d'opinion et de sentiments sur des données textuelles}

%Les textes ci-après sont extraits d'une discussion postée sur le forum de \url{www.webentrepreneurdebutant.fr}

%TODO: include graphics
% Dans les réponses des internautes, relevez de façon exhaustive toutes les tournures (grammaticales, communicatives, argumentatives, etc.) leur permettant d'exprimer leurs opinions. Classez-les par catégories, et, lorsque c'est pertinent, par polarité (positif, négatif, neutre).

  \subsection{Annotation "simple"}

 Annotez chaque mot des phrases ci-après à l'aide des étiquettes " Nom ", " Adjectif ", " Verbe ", " Pronom ", " Déterminant ", " Adverbe " ou " Autres " (notez vos annotations  sur papier ou dans un tableur, nous en aurons besoin plus tard).

\begin{itemize}
  \item La belle ferme le voile.
  \item La petite brise la glace.
\end{itemize}

\subsection{Ambiguïté grammaticale}

Annotez le paragraphe ci-dessous et comparez vos résultats avec votre voisin.
\begin{itemize}
  \item Une fois sortie de sa maison, elle se rend compte que la petite brise la glace, alors elle enfile un épais manteau.
\end{itemize}


  \section{Annotation par myriadisation (crowdsourcing)}
Pour résoudre (une partie) des problèmes d'annotation, on peut avoir recours à l'avis de plusieurs experts ou à l'avis de la foule (myriadisation).

  \subsection{Les principes du crowdsourcing}
 Pour en comprendre les buts et les principes, allez observer le travail d'Alice Millour, docteure en TAL de Sorbonne Université :
\begin{itemize}
\item \url{http://bisame.paris-sorbonne.fr/recettes/} 
\item \url{http://krik.paris-sorbonne.fr/} 
\end{itemize}
 
 Un travail d'annotation par myriadisation peut être réalisé dans tous les domaines, sur tous les types de support, par ex: \url{https://www.zooniverse.org/about}.

  \subsection{ Avantages et inconvénients du crowdsourcing}

 Que peuvent ils être ?




  \section{Evaluation des annotations}

  Si plusieurs personnes ont annoté un échantillon de façon différente, comment déterminer lequel des annotateurs a raison ?

%Pour évaluer la Cohérence intra-annotateur  comparez vos annotations sur les deux premiers exemples de l'exercice 1 avec votre voisin(e).
 Établissez maintenant la \textbf{matrice de confusion} présentant les résultats de vos deux annotations (tableau \ref{matrice}). L'annotateur 1 représente les lignes, l'annotateur 2 représente les colonnes. A l'intersection de Nom et Nom le 3 signifie que dans 3 cas, les deux annotateurs ont été d'accord sur l'étiquette nom; Sur la même ligne, le "4" signifie que dans 4 cas, annotateur 1 a étiqueté "NOM" là où l'autre annotateur a étiqueté "Pronom".


\begin{small}
\begin{table}[h]
\begin{tabular}{c|c|c|c|c|c|c|c|c}
 & Nom & Adjectif & Verbe & Pronom & Déterminant & Adverbe & Autres&Total\\
\hline
 Nom &\cellcolor{lightgray}3 &	&	&4	&	&	&&\\
\hline
Adj.&	&\cellcolor{lightgray}	&	&	&	&	&&\\ 
%\hline
Ver. 	&	&	&\cellcolor{lightgray}	&	&	&	&&\\
\hline
Pro. &	&	&	&\cellcolor{lightgray}	&	&	&\\
\hline
Dét. &	&	&	&	&\cellcolor{lightgray}	&	&\\
\hline
Adv. &	&	&	&	&	&\cellcolor{lightgray}	&\\
\hline
Aut.  &	&	&	&	&	&	&\cellcolor{lightgray}\\
\hline
\hline
Total	&	&	&	&	&	&	&\\
\end{tabular}
\caption{\label{matrice} Matrice de confusion pour l'étiquetage}
\end{table}
\end{small}
Calculez pour chaque étiquette du jeu d'annotation le pourcentage d'accord réel (somme des cases grisées*100/nombre total d'échantillons).

  \paragraph{Accord inter-annotateurs}

 La seconde étape est d'évaluer des annotations. Les échantillons utilisables (pour une analyse statistique, un classifieur automatique, etc.) sont ceux pour lesquels un grand nombre d'annotateurs s'accordent.

Calculez le pourcentage d'accord réel pour chaque étiquette. Reportez ces résultats dans un tableau récapitulatif.

 Pour vérifier que l'accord inter-annotateurs est plus élevé que le hasard, on peut mesurer le Kappa de Cohen :

$K = \frac{P_o - P_e}{1-P_e}$

Où $P_o$ est la proportion d'accord observé (somme des effectifs diagonaux divisée par la taille de l'échantillon).
Où $P_e$ est la proportion d'accord aléatoire (somme des produits des effectifs marginaux pour une même classe, divisée par le carré de la taille de l'échantillon)
%Reportez ces résultats ($P_o$, $P_e$ et K) dans un tableau récapitulatif.

Évaluez le taux d'accord entre vous et votre voisin au moyen du code ci-dessous (Notebook sur Moodle) où pour chaque annotateur on crée une liste correspondant aux étiquettes données dans l'ordre (cf listes \texttt{annotateur1} et \texttt{annotateur2} du code ci-après).
\newpage
\begin{python}
from nltk import agreement
annotateur1 = ["DET", "NOM", "ADJ", "ADJ", "ADV", "DET","ADJ", "ADV"]
annotateur2 = ["DET", "NOM", "ADJ", "NOM", "ADV", "NOM","ADV", "ADV"]
annotateurs = [annotateur1, annotateur2]
donnees = []
for i in range(len(annotateurs)):
  for j in range(len(annotateur1)):
    donnees.append([str(i), str(j), annotateurs[i][j]])
ratingtask = agreement.AnnotationTask(data=donnees)
print("kappa " +str(ratingtask.kappa()))
print("fleiss " + str(ratingtask.multi_kappa()))
print("alpha " +str(ratingtask.alpha()))
print("scotts " + str(ratingtask.pi()))
\end{python}
%Vous obtiendrez ces scores :
%Fleiss Kappa for 3 raters = x (le Kappa) SE = x (l'écart-type)
%95%CI = x to x (l'intervalle de confiance)

On peut désormais représenter cela sous forme d'une matrice de confusion :

\begin{python}
import seaborn as sns
import matplotlib.pyplot as plt
from sklearn.metrics import confusion_matrix
cm = confusion_matrix(annotateur1, annotateur2)
classes = sorted(list(set(annotateur1+annotateur2)))
fig, ax = plt.subplots(figsize=(10,10))
sns.heatmap(cm, 
	annot=True, 
	xticklabels=classes, 
	yticklabels=classes,
	cmap=plt.cm.Reds )
plt.ylim([len(classes), 0])
plt.show()
\end{python}
  \subsection{Finalisation}% et évaluation des scores (à finir à la maison et à rendre en Binôme)}
Vous rendrez un notebook Python qui compilera vos résultats. Tout d'abord, annotez chacun de votre côté l'exemple suivant :
 "Je me promène dans la campagne, au milieu des champs. J'arrive vers la magnifique ferme de mon voisin, et je remonte le chemin afin d'aller admirer son potager. Cependant, arrivé à la clôture, je me rends compte que la belle ferme le voile."

Calculez les scores $P_o$, $P_e$ et K puis répondez aux questions suivantes:
\begin{itemize}
  \item Représentez vos annotations avec une matrice de confusion
  \item Quelle étiquette présente le meilleur taux d'accord ?
%  \item Votre propre annotation est-elle fiable ? Celle des autres juges également ?
  \item Les données sont-elles homogènes ? Y a-t-il des classes sous-/sur-représentées ?
  \item De quelle façon pourrions-nous améliorer les scores d'accord?
\end{itemize}

%\textcolor{blue}{\textbf{Vous rendrez ceci sur Moodle pour le 08/12/2019}}


\noindent\fcolorbox{red}{lightgray}{
\begin{minipage}{15cm}
\subsection*{Approfondissement}
 Consultez l'article suivant pour confronter vos idées avec celles d'une chercheuse du domaine :

\url{https://www.liberation.fr/societe/2015/05/07/miracles-et-mirages-du-crowdsourcing_1297262}

\end{minipage}
}
