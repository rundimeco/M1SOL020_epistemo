
\newcommand{\numTD}{TD6}
\newcommand{\themeTD}{Logique}

\begin{center}
\begin{tabular}{|p{2cm}c|}
\hline
{\includegraphics[width=1.8cm,viewport=0 0 337 248]{../images/sorbonne.png}} & \raisebox{2ex}{\begin{Large}\textbf{M1SOL020}\end{Large}}\\
2019-2020& \raisebox{2ex}{\begin{Large}\textbf{ Epistémologie de l'Informatique}\end{Large}}\\
&  \begin{large}\textbf{\numTD}\end{large} \\
&  \begin{large} \textbf{\themeTD}\end{large} \\
& Gaël Lejeune, Sorbonne Université \\
& \tiny{Inspiré d'Agnès Delaborde 2015-2016}\\
\hline
\end{tabular}
\end{center}



\hrule
%%%%%%%%%%%%%%%%%%%%%%%%%EN-TETE%%%%%%%%%%%%%%%%%%%%%%%%%%%%%
%\renewcommand{\contentsname}{Sommaire du TD}
%\tableofcontents

\noindent\fcolorbox{red}{lightgray}{
\begin{minipage}{12cm}
\section*{Objectifs}

\begin{itemize}
  \item Savoir exploiter les principes de logique dans les structures itératives
  \item Savoir analyser une expression logique
\end{itemize}
\end{minipage}
}

\section{Itérations}

\textcolor{blue}{\bf{Autant que possible, n'exécutez pas immédiatement le code \textsc{Python}} de cet exercice, il s'agit de réfléchir "sur papier" avant de lancer}. Le notebook avec le code est sur le Moodle.

\begin{enumerate}
  \item A quel moment une boucle s'interrompt-elle ?
  \item Dans l’exemple suivant, à partir de quand la condition d'entrée devient fausse ?

\begin{python}
i = 2
while i<=8:
  print("coucou")
  print(i)
  i+=1
print("Fin")
\end{python}

  \item La finalité de l'exemple ci-dessous est-elle similaire à l'exemple précédent ?
\begin{python}
for i in range(2, 9):
  print("coucou")
  print(i)
  i+=1
print("Fin")
\end{python}
  \item Analysez les différentes propositions de l'exemple ci-dessous et indiquez quelle est la condition de sortie de boucle.
\begin{python}
i = 5
encore = True
while encore == True:
  print("coucou")
  i+=1
  if i>10:
    encore = False
print("Fin")
\end{python}

  \item Étudiez la proposition logique utilisée dans cet algorithme. Que va faire le code ?

\begin{python}
if True:
  print("c'est vrai")
\end{python}

  \item Et dans ce cas ?
\begin{python}
if False:
  print("c'est faux")
\end{python}

  \item Que va faire ce code ?
\begin{python}
while True:
  print("c'est vrai")
\end{python}

  \item L'algorithme ci-dessous permet-il d'écrire "bonjour" ?
\begin{python}
A = True
B = False
if A and B:
  print("bonjour")
\end{python}
  \item Dans l'exemple ci-dessous, pour quelle valeur de i pourrons-nous sortir de la boucle ?
\begin{python}
A = True
B = False
i = 0
while A or B:
  print("bonjour")
  if i>10 and i<20:
    A = False
  i+=1 
\end{python}

  \item Étudiez le comportement de l'algorithme suivant. Peut-on rentrer dans la boucle ? Quand sortirons nous de la boucle ?
\begin{python}
A = True
B = True
i = 0
while A or B:
  print("bonjour")
  A = False
\end{python}

  \item Quelle différence avec celui-ci (NB: \^{} est le ou exclusif ou "XOR")?
\begin{python}
A = True
B = True
i = 0
while A ^ B:
  print("bonjour")
  A = False
\end{python}

\end{enumerate}

%\section{Analyse de propositions et tables de vérité}

  \section{Apprendre les opérateurs, compléter des tables de vérité}

\subsection{Négation logique $\neg$}


\begin{tabular}{c|c}

A	&$\neg$ A\\
\hline
VRAI	&\\
\hline
FAUX	&\\
\end{tabular}

\subsection{ET logique ($\wedge $),  OU logique ($\vee $)}

\begin{tabular}{c|c|c||c|c|c}
%  \multicolumn{3}{c}{ET logique ($\wedge $)}
%& \multicolumn{3}{c}{OU logique ($\vee $)} \\
%\hline
A	&B	&A$\wedge$B&A	&B	&A $\vee$ B\\
\hline
\hline
VRAI	&VRAI	&	&VRAI	&VRAI	&\\
\hline
VRAI	&FAUX	&	&VRAI	&FAUX	&\\
\hline
FAUX	&VRAI	&	&FAUX	&VRAI	&\\
\hline
FAUX	&FAUX	&	&FAUX	&FAUX	&\\
\end{tabular}

\subsection{XOR (OU-exclusif, $\oplus$)}%, Implication ($\Rightarrow$)}

\begin{tabular}{c|c|c|}%|c|c|c}
%  \multicolumn{3}{c}{ET logique ($\wedge $)}
%& \multicolumn{3}{c}{OU logique ($\vee $)} \\
%\hline
A	&B	&A$\oplus$B\\%&A	&B	\\%&A $\Rightarrow$ B\\
\hline
\hline
VRAI	&VRAI	&	\\%&VRAI	&VRAI	&\\
\hline
VRAI	&FAUX	&	\\%&VRAI	&FAUX	&\\
\hline
FAUX	&VRAI	&	\\%&FAUX	&VRAI	&\\
\hline
FAUX	&FAUX	&	\\%&FAUX	&FAUX	&\\
\end{tabular}


  \subsection{Compléter des tables de vérité complexes}
% ET \wedge$ 		OU $\vee$ B
% OU-exclusif, $\oplus$ Implication ($\Rightarrow$)

\begin{table}[h]
\begin{tabular}{c|c|c}
A	&B	& $\neg$ A $\vee$ $\neg$ B\\ 
\hline
\hline
VRAI	&VRAI	&	\\
\hline
VRAI	&FAUX	&	\\
\hline
FAUX	&VRAI	&	\\
\hline
FAUX	&FAUX	&	\\
\end{tabular}
%\caption{Double négation}
\end{table}

\begin{table}[h]
\begin{tabular}{c|c|c|c|c}
A	&B	& A $\wedge \neg$ B&B $\vee \neg$ A
		&(A $\wedge \neg$ B) $\vee$ (B $\vee \neg$A)\\
\hline
\hline
VRAI	&VRAI	&	&&\\
\hline
VRAI	&FAUX	&	&&\\
\hline
FAUX	&VRAI	&	&&\\
\hline
FAUX	&FAUX	&	&&\\
\end{tabular}
%\caption{Imbrication}
\end{table}

\begin{table}[h]
\begin{tabular}{c|c|c|c|c|c}
A	&B	&C	&(A$\wedge$B)	&$\neg$C & (A$\wedge$B) $\vee$ $\neg$C\\
\hline
\hline
VRAI	&$\neg$A&$\neg$B&		&	 &\\
\hline
VRAI	&FAUX	&$\neg$A&		&	 &\\
\hline
FAUX	&VRAI	&$\neg$B&		&	 &\\
\hline
FAUX	&FAUX	&$\neg$B&		&	 &\\
\end{tabular}
%\caption{Dépendances}
\end{table}

\end{document}

%TODO: réfléchir à rendre ce problème utilisable
  \item Considérez ce problème algorithmique :
Soient :
la variable Chaîne (une chaîne de caractères),
une liste nommée Liste contenant des chaînes de caractères,
un booléen nommé Bool
Vous désirez réaliser un algorithme qui n’exécutera PAS un code UNIQUEMENT dans ce cas :
Si Bool est faux, et que Chaîne fait partie de Liste.
Dans tous les autres cas, le code devra s’exécuter.
Ne rédigez pas l’algorithme, mais seulement la table de vérité associée. A quelle table de vérité vue ci-dessus cela correspond-il ?
  \item Analysez les propositions ci-dessous. Pour chaque valeur de A, B et C telles qu’indiquées, indiquez si la proposition sera vraie ou fausse.
A
B
(A ∧ ¬ B) ∨ (B ∧ ¬ A)
VRAI
VRAI

FAUX
VRAI

VRAI
FAUX

FAUX
FAUX

Etudiez le tableau ci-dessus : à quelle table de vérité cela vous fait penser ?

A
B
¬A ∨ ¬B
VRAI
VRAI

FAUX
VRAI

VRAI
FAUX

FAUX
FAUX


A
B
(A ⇒ B) ∧ ¬B
VRAI
VRAI

FAUX
VRAI

VRAI
FAUX

FAUX
FAUX


A
B
C
(A ∧ B) ∨ ¬C
VRAI
¬A
¬B

FAUX
VRAI
¬A

VRAI
FAUX
¬B

FAUX
FAUX
¬B


