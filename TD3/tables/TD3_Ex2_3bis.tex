
\begin{tabular}{l|l|l|l|l|l|l|l|l|l}
Binaire		&   0	&    0	&    0	&    0	&    0	&    \textbf{1}	&    0	&    0	&    \textbf{1}	\\
\hline
Décomposition	&$1*2^9$ & $1*2^8 $ &$1*2^7$ & $1*2^6$ & $1*2^5$ & \textbf{$1*2^4$} & $1*2^3$ & $1*2^1$ & $1*2^0$\\
\hline
Valeur décimale	&256	&128	&64	&32	&16	&\textbf{8}	&4	&2	&\textbf{1}\\	
%\hline
%Valeurs utilisées	&0	&0	&0	&0	&0	&\textbf{8}	&0	&0	&\textbf{1}\\
\end{tabular}
\vspace{0.2cm}


\textit{Lecture : Si je veux représenter 9 je dois trouver la somme correspondante dans la ligne des valeurs décimales. En l'occurrence, c'est 8 + 1. Je peux donc remonter à la ligne Binaire pour récupérer les "1" dont j'ai besoin, le reste ce sont des "0 : $8_{\text{10}}$ = $000001001_{\text{12}}$ }
%Je vois dans la ligne des "chiffres en binaire" que j'obtiens 000001000. Je peux retirer les premiers zéros qui ne sont pas nécessaires dans cet exercice, et dire donc que 8(10) = 1000(2).
